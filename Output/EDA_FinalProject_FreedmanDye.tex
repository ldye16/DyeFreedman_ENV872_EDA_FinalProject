% Options for packages loaded elsewhere
\PassOptionsToPackage{unicode}{hyperref}
\PassOptionsToPackage{hyphens}{url}
%
\documentclass[
  12pt,
]{article}
\usepackage{amsmath,amssymb}
\usepackage{lmodern}
\usepackage{iftex}
\ifPDFTeX
  \usepackage[T1]{fontenc}
  \usepackage[utf8]{inputenc}
  \usepackage{textcomp} % provide euro and other symbols
\else % if luatex or xetex
  \usepackage{unicode-math}
  \defaultfontfeatures{Scale=MatchLowercase}
  \defaultfontfeatures[\rmfamily]{Ligatures=TeX,Scale=1}
  \setmainfont[]{Times New Roman}
\fi
% Use upquote if available, for straight quotes in verbatim environments
\IfFileExists{upquote.sty}{\usepackage{upquote}}{}
\IfFileExists{microtype.sty}{% use microtype if available
  \usepackage[]{microtype}
  \UseMicrotypeSet[protrusion]{basicmath} % disable protrusion for tt fonts
}{}
\makeatletter
\@ifundefined{KOMAClassName}{% if non-KOMA class
  \IfFileExists{parskip.sty}{%
    \usepackage{parskip}
  }{% else
    \setlength{\parindent}{0pt}
    \setlength{\parskip}{6pt plus 2pt minus 1pt}}
}{% if KOMA class
  \KOMAoptions{parskip=half}}
\makeatother
\usepackage{xcolor}
\usepackage[margin=2.54cm]{geometry}
\usepackage{color}
\usepackage{fancyvrb}
\newcommand{\VerbBar}{|}
\newcommand{\VERB}{\Verb[commandchars=\\\{\}]}
\DefineVerbatimEnvironment{Highlighting}{Verbatim}{commandchars=\\\{\}}
% Add ',fontsize=\small' for more characters per line
\usepackage{framed}
\definecolor{shadecolor}{RGB}{248,248,248}
\newenvironment{Shaded}{\begin{snugshade}}{\end{snugshade}}
\newcommand{\AlertTok}[1]{\textcolor[rgb]{0.94,0.16,0.16}{#1}}
\newcommand{\AnnotationTok}[1]{\textcolor[rgb]{0.56,0.35,0.01}{\textbf{\textit{#1}}}}
\newcommand{\AttributeTok}[1]{\textcolor[rgb]{0.77,0.63,0.00}{#1}}
\newcommand{\BaseNTok}[1]{\textcolor[rgb]{0.00,0.00,0.81}{#1}}
\newcommand{\BuiltInTok}[1]{#1}
\newcommand{\CharTok}[1]{\textcolor[rgb]{0.31,0.60,0.02}{#1}}
\newcommand{\CommentTok}[1]{\textcolor[rgb]{0.56,0.35,0.01}{\textit{#1}}}
\newcommand{\CommentVarTok}[1]{\textcolor[rgb]{0.56,0.35,0.01}{\textbf{\textit{#1}}}}
\newcommand{\ConstantTok}[1]{\textcolor[rgb]{0.00,0.00,0.00}{#1}}
\newcommand{\ControlFlowTok}[1]{\textcolor[rgb]{0.13,0.29,0.53}{\textbf{#1}}}
\newcommand{\DataTypeTok}[1]{\textcolor[rgb]{0.13,0.29,0.53}{#1}}
\newcommand{\DecValTok}[1]{\textcolor[rgb]{0.00,0.00,0.81}{#1}}
\newcommand{\DocumentationTok}[1]{\textcolor[rgb]{0.56,0.35,0.01}{\textbf{\textit{#1}}}}
\newcommand{\ErrorTok}[1]{\textcolor[rgb]{0.64,0.00,0.00}{\textbf{#1}}}
\newcommand{\ExtensionTok}[1]{#1}
\newcommand{\FloatTok}[1]{\textcolor[rgb]{0.00,0.00,0.81}{#1}}
\newcommand{\FunctionTok}[1]{\textcolor[rgb]{0.00,0.00,0.00}{#1}}
\newcommand{\ImportTok}[1]{#1}
\newcommand{\InformationTok}[1]{\textcolor[rgb]{0.56,0.35,0.01}{\textbf{\textit{#1}}}}
\newcommand{\KeywordTok}[1]{\textcolor[rgb]{0.13,0.29,0.53}{\textbf{#1}}}
\newcommand{\NormalTok}[1]{#1}
\newcommand{\OperatorTok}[1]{\textcolor[rgb]{0.81,0.36,0.00}{\textbf{#1}}}
\newcommand{\OtherTok}[1]{\textcolor[rgb]{0.56,0.35,0.01}{#1}}
\newcommand{\PreprocessorTok}[1]{\textcolor[rgb]{0.56,0.35,0.01}{\textit{#1}}}
\newcommand{\RegionMarkerTok}[1]{#1}
\newcommand{\SpecialCharTok}[1]{\textcolor[rgb]{0.00,0.00,0.00}{#1}}
\newcommand{\SpecialStringTok}[1]{\textcolor[rgb]{0.31,0.60,0.02}{#1}}
\newcommand{\StringTok}[1]{\textcolor[rgb]{0.31,0.60,0.02}{#1}}
\newcommand{\VariableTok}[1]{\textcolor[rgb]{0.00,0.00,0.00}{#1}}
\newcommand{\VerbatimStringTok}[1]{\textcolor[rgb]{0.31,0.60,0.02}{#1}}
\newcommand{\WarningTok}[1]{\textcolor[rgb]{0.56,0.35,0.01}{\textbf{\textit{#1}}}}
\usepackage{graphicx}
\makeatletter
\def\maxwidth{\ifdim\Gin@nat@width>\linewidth\linewidth\else\Gin@nat@width\fi}
\def\maxheight{\ifdim\Gin@nat@height>\textheight\textheight\else\Gin@nat@height\fi}
\makeatother
% Scale images if necessary, so that they will not overflow the page
% margins by default, and it is still possible to overwrite the defaults
% using explicit options in \includegraphics[width, height, ...]{}
\setkeys{Gin}{width=\maxwidth,height=\maxheight,keepaspectratio}
% Set default figure placement to htbp
\makeatletter
\def\fps@figure{htbp}
\makeatother
\setlength{\emergencystretch}{3em} % prevent overfull lines
\providecommand{\tightlist}{%
  \setlength{\itemsep}{0pt}\setlength{\parskip}{0pt}}
\setcounter{secnumdepth}{5}
\ifLuaTeX
  \usepackage{selnolig}  % disable illegal ligatures
\fi
\IfFileExists{bookmark.sty}{\usepackage{bookmark}}{\usepackage{hyperref}}
\IfFileExists{xurl.sty}{\usepackage{xurl}}{} % add URL line breaks if available
\urlstyle{same} % disable monospaced font for URLs
\hypersetup{
  pdftitle={Pika Distribution and Trends at Niwot Ridge},
  pdfauthor={Jacob Freedman and Logan Dye},
  hidelinks,
  pdfcreator={LaTeX via pandoc}}

\title{Pika Distribution and Trends at Niwot Ridge}
\usepackage{etoolbox}
\makeatletter
\providecommand{\subtitle}[1]{% add subtitle to \maketitle
  \apptocmd{\@title}{\par {\large #1 \par}}{}{}
}
\makeatother
\subtitle{\url{https://github.com/ldye16/DyeFreedman_ENV872_EDA_FinalProject}}
\author{Jacob Freedman and Logan Dye}
\date{}

\begin{document}
\maketitle

\newpage
\tableofcontents 
\newpage
\listoftables 
\newpage
\listoffigures 
\newpage

\begin{Shaded}
\begin{Highlighting}[]
\CommentTok{\# Load your datasets}
\NormalTok{Pika\_Raw }\OtherTok{\textless{}{-}} \FunctionTok{read.csv}\NormalTok{(}\StringTok{"./Data/Raw/pika\_demography.cr.data.csv"}\NormalTok{,}
    \AttributeTok{stringsAsFactors =} \ConstantTok{TRUE}\NormalTok{)}
\NormalTok{Climate\_Raw }\OtherTok{\textless{}{-}} \FunctionTok{read.csv}\NormalTok{(}\StringTok{"./Data/Raw/d{-}1cr23x{-}cr1000.daily.ml.data.csv"}\NormalTok{,}
    \AttributeTok{stringsAsFactors =} \ConstantTok{TRUE}\NormalTok{)}
\end{Highlighting}
\end{Shaded}

\hypertarget{rationale-and-research-questions}{%
\section{Rationale and Research
Questions}\label{rationale-and-research-questions}}

The American Pika is a threatened small mammal endemic to alpine tundra
habitat in the Rocky Mountains and Sierra Nevadas. Climate change is a
growing threat worldwide, and is expected to have a particular impact on
high elevation habitats. Pikas are sensitive to both temperature and
changes in the water balance (impacted by seasonality of snowmelt), so
climate change could undoubtedly threaten their long-term viability. In
addition, as Pikas occupy habitat at the tops of mountain ridges, their
possible migration either northward or to higher ridges would involve
them crossing over lower elevation valleys. Their ability to do so is
unknown, which makes their possibility of survival even murkier.

Niwot Ridge is one of the most heavily studied alpine areas in the Rocky
Mountains. It is home to multiple research projects, as it is both a
NEON and LTER Site. LTER researchers have gathered data on Pika
demography and climate over the past 10-20 years. We used multiple
public datasets to answer multiple questions regarding trends in climate
and Pika populations at Niwot Ridge.

\begin{enumerate}
\def\labelenumi{\arabic{enumi}.}
\tightlist
\item
  Is temperature changing over time at Niwot Ridge?
\item
  Are Pika populations changing over time and is there a correlation
  between population changes and temperature changes?
\item
  Are Pika populations changing spatially over time? Are they seeking
  higher elevation sites?
\item
  An increase in zoonotic diseases is a well documented effect of
  warming temperatures. Is mite and flea prevalence in Pikas changing
  over time and with temperature?
\end{enumerate}

\newpage

\hypertarget{dataset-information-and-wrangling}{%
\section{Dataset Information and
Wrangling}\label{dataset-information-and-wrangling}}

Both climate and pika data was sourced from the Niwot Ridge LTER
website. All datasets are for public use and are available here:
\url{https://nwt.lternet.edu/data-catalog} Significant wrangling was
required to answer the research questions. The general process is
explained below for each research question.

Temperature changes over time:

Climate data was available from 2000 to 2021, but we selected for years
in which there was an overlap of available temperature data with the
Pika data (2008 to 2018). The dataset contained the site and device
names, daily min, max, and avg air temperature, relative humidity,
barometric pressure, wind speed/direction, solar radiation and soil
temperature. We then used the package ``zoo'' to interpolate NA values
in the temperature data.

Pika population changes over time and in relation to temperature:

Pika data was available from 2008 to 2020, but we selected for years in
which there was an overlap of available Pika data with the temperature
data (2008 to 2018). The dataset contained the site, data, slope aspect,
location of capture, identification information, demography data,
biological samples collected and pest presence. In order to estimate
population over time, we identified annual Pika captures at Niwot Ridge.
This required eliminating any possibility of recounts within the same
year. We first filtered out any captures that did not have a ``tag
type'', meaning that they were not tagged at all and not enough
information is provided to rule it out as a recapture. Second, we
filtered out observations where the tag type was recorded, but for an
unknown reason none of the tag information (ear tag color or number
code) was recorded. Finally, we split the dataset into separate datasets
for each year and found the number of unique tag IDs to derive annual
counts.

To prepare for our temperature and pika abundance regressions, we needed
to calculate mean annual temperature values. We took the interpolated
temperature dataset and grouped by year, then summarized by average
daily temperature to calculate annual means.

Pika spatial distribution changes:

To assess spatial changes, observations without spatial attributes
(easting, northing) were eliminated. Mean annual pika locations were
determined by summarizing the easting and northing columns. Both
individual and mean annual locations were converted to spatial data
frames using the NAD83 Zone 13N projected coordinate system.

Parasite Data:

To determine if there was a change in parasite abundance over time and
with changing temperatures, the flea and ear mite data first needed to
be isolated from the total pika demography data. The flea and ear mite
data within the demography data are flea observations, fleas sampled,
ear mite observations, and ear mites sampled. For both flea observations
and fleas sampled, the recorded value equates to the number of
individual fleas either observed on or sampled from the individual pika.
However, for ear mites sampled, the recorded values are binary. A ``0''
is recorded for individual pikas that did not have ear mites sampled,
and a ``1'' is recorded for pikas that were sampled. For ear mites
observed, the data is represented in categories. The categories are
``N'' for none observed, ``L'' for low density, ``M'' for medium
density, ``H'' for high density, ``NA'' for individuals whose data was
not recorded in the main trapping notebook, and ``NS'' for Pikas who
were not sampled for mites in this way. There are also a few instances
of either a ``0'' or a ``1'' being recorded and a single recording of
``N?''.

Once we had isolated the flea and ear mite data from the demography
data, we split the dataframe into two, one with only flea data and one
with only ear mite data. For the flea data, we grouped by PikaID to
isolate individual pikas that were recaptured in a given year. Next, we
split the data by year and found the mean value of fleas sampled and
fleas observed per pika for each year. We then merged the single-year
dataframes to create one dataframe with the average number of fleas
observed and sampled per pika per year. We wrangled the mite sampled
data the same way. However, since the data was binary, either ``0'' or
``1'', the resulting ear mite sampled column represented a proportion of
captured pikas sampled for ear mites. Generally, if ear mites were
observed on a pika, they were sampled. This means the proportion of
pikas sampled for ear mites represents the proportion of total pikas
captured in a given year with ear mites present.

For ear mite observed data, we used a different wrangling technique to
account for the categorical data. The NA, NS, 0, 1, and N? observations
were all dropped. We retained only observations where ear mite
observation data was collected. These are N, L, M, and H. Once we had
selected the variables we wanted, we grouped them by year and
observation type and summed the amount of each observation variable
recorded for a year. This gave us a total count for each observation
variable for each year. To account for the different number of pika
captures each year, we converted the data from a count total to the
proportion of each categorical variable represented each year.

In the final step, we joined the wrangled flea and mite dataframes into
one final parasite dataframe. The final parasite dataframe was then
joined to the climate data, and we had our final merged parasite and
climate dataframe.

\newpage

\hypertarget{exploratory-analysis}{%
\section{Exploratory Analysis}\label{exploratory-analysis}}

\newpage

\hypertarget{analysis}{%
\section{Analysis}\label{analysis}}

\hypertarget{question-1-is-temperature-changing-over-time-at-niwot-ridge}{%
\subsection{Question 1: Is temperature changing over time at Niwot
Ridge?}\label{question-1-is-temperature-changing-over-time-at-niwot-ridge}}

\hypertarget{question-2-are-pika-populations-changing-over-time-and-is-there-a-correlation-between-population-changes-and-temperature-changes}{%
\subsection{Question 2: Are Pika populations changing over time and is
there a correlation between population changes and temperature
changes?}\label{question-2-are-pika-populations-changing-over-time-and-is-there-a-correlation-between-population-changes-and-temperature-changes}}

\#\#Question 3: Are Pika populations changing spatially over time? Are
they seeking higher elevation sites?

\#\#Question 4: One well documented effect of warming temperatures is an
increase in zoonotic diseases. Is mite and flea prevalence in Pikas
changing over time and with temperature?

To analyze this question, linear regressions were ran on flea and mite
data over both time and temperature.

\newpage

\hypertarget{summary-and-conclusions}{%
\section{Summary and Conclusions}\label{summary-and-conclusions}}

\newpage

\hypertarget{references}{%
\section{References}\label{references}}

\textless add references here if relevant, otherwise delete this
section\textgreater{}

\end{document}
